\begin{abstract}
The positive effect of multiple mobile data collectors (or Mobile 
Ubiquitous LAN Extensions or MULEs) on network lifetime and energy 
efficiency of a sparse sensor network, when used in data collection from 
them, is well known. However, the increased latency (due to physical 
movement of a robot instead of radio communication) may be a problem in 
some latency sensitive applications. Motivated by this problem, many 
MULE path scheduling problems in attempts to decrese latency 
employ multiple MULEs and long range relay nodes for data collection 
from the sensor network. Most of the work 
use some kind of network partitioning schemes that do not make use of 
rendezvous of two MULEs. To overcome this problem, some of these works 
use long range relay nodes or a common point of origin of the data 
collection tours of all the employed MULEs. In this thesis we 
consider latency bound data collection from a sparse sensor network 
using only MULEs as the means to eatablish connectivity, and also 
alleviate the problems mentioned above. The sensors in the field are 
grouped under planar circular discs of radius equal to their sensing 
range using disc cover problem. The centers of these discs are called 
"location nodes". In other words, locations nodes define the positions
where a MULE comes to gather data from the sensors covered by it. The 
latency bound, given as an input, determines 
the bound on the path length of a MULE. Using this bound, we propose a 
MULE path selection heuristic, which divides the movement and data 
collection work among multiple MULEs, each of whose data collection tour 
respects the bound. The tours are chosen in such a way that the MULEs 
which are supposed to rendezvous for forwarding data, have a definite 
rendezvous point in the plane to do the same.

\end{abstract}