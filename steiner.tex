\chapter{Path selection} \label{chap:steiner}

%\section{Subproblems}

Our approach to a better heuristics
is driven by the concept of location graph and using the same for
construction of low latency tours of the MULEs in data collection phase. 
Two critical subproblems encountered in this context are: 
(i) Euclidean Minimum Stiener Tree (EMST), and (ii)  Euclidean Travelling Saleman Problem (ETSP). Both the problems have already been introduced
in chapter~\ref{chapter2}. We will briefly restate these here for
sake of completeness and convenience in description of our heuristics.

%%%%%%%%%%%%%%%%%%%%%%%%%%%%%%%%%%%%%%%%%%%%%
\begin{definition}[Euclidean Travelling Salesman Problem]
Given a set of n 2D points in a plane find a minimum cost tour of all
the points, visiting every point exactly once.
\end{definition}
We use Christofides algorithm~\cite{christofides} as the approximation 
heuristic for computing TSP route of a set of points.

{\bf Comment:} I believe these subproblem are independent concepts. 
So, these should be introduced in chapter 2 little more elaborately
with adequate nontrivial examples.


\begin{definition}[Euclidean minimum Steiner Tree problem]
Given a set $P$ of points in a 2-D plane as input, the output is a network of line segements connecting all of the points in $S$, with the smallest total (Euclidean) length.
\end{definition}
The line segments making the Steiner Tree need just be incident on the 
points in $S$. This implies that the algorithm is free to use additional 
points from the plane, if necessary, to produce the smallest total length 
network. The additional points are called {\em Steiner Points}.
%%%%%%%%%%%%%%%%%%%%%%%%%%%%%%%%%%%%%%%%%%%%%%%%%%%%%%%%%%%

The location nodes (the nodes in location graph) computed earlier, form 
the set $P$, and the EMST is generated using the set $P$.
The pseudo code for the heuristic
which appear later in section~\ref{sec:heuristicAlgo} this chapter use 
the EMST as input. The choice of EMST as data structure over MST is 
guided by the fact that the goal of our heuristic to divide the set of location nodes into multiple sets with approximately the same TSP time units.
The TSP time unit is the time taken by a MULE to visit all points on EMST.
the cost of ETSP on a set is atmost twice the cost of the EMST on it. Therefore, it should be reasonable to use the Steiner tree traversal as a guide for partitioning. Furthermore, in the case where the field also includes obstacles, Steiner trees lend themselves naturally to cover all the points due the properties of Steiner points~\cite{oaest99} as explained in 
section~\ref{sec:steinerPoint} of chatper~2. Though we plan not to cover the obstacle avoidance case, we briefly sketch the underlying ideas in 
Chapter~\ref{chap:concl}.
For computing Steiner tress, we use the exact solution finder software 
GeoSteiner~\cite{geosteiner1}~\cite{geosteiner2}~\cite{geosteiner3}. 

{\bf Comment:} Here you could show a Steiner tree generated by
GeoSteiner. 

\section{Path Selection Heuristic}

%\subsection{Aim}
The aim of path selection heuristic is to find a minimum partitioning of
the set of location nodes of a location graph by addition of 
extra Steiner points such that following conditions are satisfied.

\begin{itemize}
\item Each of the subsets has a TSP tour length (in units of time) less than a per-specified value $L$, and for any two sets $S_{i}$ and $S_{j}$, $S_{i} \cap S_{j} \le 1$.
\item Let $V$ be the set of all subsets $S_{i}$. Let $E$ be the set of pairs of subsets $(S_{i},S_{j})$ such that $S_{i} \cap S_{j} = 1$. Then the graph $G(V,E)$ should be a connected graph.
\end{itemize}

It assumed that the time a MULE spends in a network while collecting data
consists of three main components: (i) travelling from one location node to another $t_{TSP}$, (ii) talking to sensors belonging to a location node $t_{LS}$, (iii) talking to other MULEs/Base station $t_{MBS}$. Obviously, the 
contribution due to last two components of a MULE's operation remains
the same for in any subset of location nodes assigned to the same MULE. 
Furthermore, both $t_{LS}$ and $t_{MBS}$ are small compared to $t_{TSP}$. 

We first describe the algorithm for a simplified heuristic that ignores the last two components of a MULE's, namely, $t_{LS}$ and $t_{MBS}$. Subsequently, we modify the above algorithm to take into account the second component of time $t_{LS}$. We assume that MULE to MULE data transfer times are shorterdue to two reasons, namely, (i) MULEs may use data aggregation while sending data to fellow a MULE or BS, and (ii) MULEs being relative expensive and more robust than sensor node, typically have higher bandwidth for inter MULE data transfer. The component $t_{LS}$ for each location node can be given as an input, or observed before running the heuristic, by simply sending one MULE on a tour of all the location nodes in the field to measure and record such times beforehand.


\subsection{Heuristics}

The overall strategy is to create an EMST of the location nodes and then divide this tree into subgraphs, using the tree edges as the guide. Each subgraph's set of nodes will be covered by one MULE (henceforth, this set of nodes will be called a subtour). An edge of the tree is said to have weight equal to its length divided by the speed of the MULE (time taken by the MULE to cover that edge). The weight of a location node is equal to the time a MULE has to wait there for data collection (called pause time). It depends on the latency bound and the number of sensor nodes covered by that location node. Steiner points have zero weight.

Consider any euclidean spanning tree of a set of 2D points in a plane (none of the points have weights). Clearly, one way to visit all points would be to start from the root, and visit the nodes in the depth first search order, always travelling along the edges. This would take time equal to twise the weight of the tree (each edge travelled twice, once for going from parent to child, and once for coming back to the parent from the child). Thus, the optimal travelling salesman tour must be bound by twice the weight of the tree.

We now describe the algorithm that will not take in to account $t_{LS}$,and will be used to generate simulation results. Given any tree $T$, we start from the given root node $root$. $root$ then becomes the current node $curr$. Then following steps of Prim's algorithm~\cite{Prim}, we first mark $curr$ as visited, then we insert all the incident edges of the current node with unvisited nodes to the min-heap $edgeHeap$. Computation of a tour for a MULE consists of two stages coded in two inner while loops.

In the first stage, represented by the first inner while loop, every time we an edge $(v1,v2)$ is selected and delete from $edgeHeap$, all the edges incident to $v1$ which have unvisited end nodes are inserted into the heap. Then, $v1$ is inserted to $currSet$. We keep popping edges from $edgeHeap$ until either $edgeHeap$ becomes empty or, adding any more edges to $currSet$ from $edgeHeap$ leads to total weight exceeding $\frac{L}{hApprox}$. Note that the second condition ensures that the TSP weight of our points set $currSet$ does not exceed $L$. The bound for $L_{hApprox}$ depends completely upon the heuristic used for TSP problem, which is approximation ratio of the heuristic used for TSP, $\times$ 2 ({\bf comment: what is it?}).

Any Steiner point whose all adjacent nodes belong to same subtour will useless. Because, such a Steiner point neither serves as a connecting node between different subtours, nor represents a location node. So, such a Steiner point should be eliminated form the subtour. Function $cleanTour$ is used precisely for this purpose. {\bf comments: what does the next sentence supposed to mean?} Every time we test the inclusion of a new edge in the second stage.

Actual calculation of TSP of the selected node set is performed in the second stage while  %we calculate the actual TSP of our point set and 
continuing popping of edges from $edgeHeap$ and adding these to $currSet$. Before adding a node to $currSet$, we test whether its inclusion will make $currT$ exceed $L$ or not. If yes, then $currSet$ is the final node set for this MULE. Otherwise, the second inner while loop repeats. %until hen our $currSet$ for this MULE is final. 
The edge records still left in the Heap, after completion of one full iteration of the second inner while loop, form the boundary of the nodes in $currSet$. We call this set as $i$th after completion of $i$th iteration of the loop. %  with respect to the Steiner tree, 
These nodes are pushed in the $boundary$, from where the next $root$ for the next MULE's is chosen.% point set to cover, 


\begin{algorithm}
\caption{Dividing the set nodes of a given Steiner tree into subsets of bounded TSP time}\label{euclid}
\begin{algorithmic}
\Function{greedySteiner}{A Steiner tree $T$ of location in a plane, An array $w$ of the number of sensors under a location node, Starting vertex $root$, desired upper bound on latency $L$}\Comment{this function returns S: Set of tours, each with touring time $\le$ L}
\State Set  $S$
\State $N \gets$ number of vertices in T
\State $hApprox \gets 3.0$
\State $MSPEED \gets$ speed of the MULE used
\State $SDT \gets$ sensor data throughput
\State $SSR \gets$ sensor data sampling rate
\State $ASP \gets$ $\frac{(L\times SSR)}{SDT}$ (average sensor pause time for one sensor)
\State $queue$ boundary.push($root$) 
\State $bool$ $visited[N]$
\State $vertex$ curr 
\For{$i \gets 1,N$}
	\State $visited[i] \gets false$ 
\EndFor
\While{1}
	\If{boundary.empty()}
		\State break
	\EndIf
	\State $currT \gets 0.0$ 
	\State $cycleWeight \gets 0.0$ 
	\State Min\_heap $edgeHeap$  
	\State $curr \gets root$ 
	\State Set $currSet$, $tempSet$ 
	\State $visited[curr] \gets true$ 
	\State Set $U \gets$ all unvisited vertices adjacent to $cur$ 
	\ForAll{ vertex $v$ in U}
		\State $edgeHeap.push((hApprox \times dist(v,curr)) + (ASP\times w[curr]) , (v,curr))$ 
	\EndFor

	\State $currSet$.insert($curr$) 
	\While{$\neg$ edgeHeap.empty()}
		\State $nextWeight \gets edgeHeap$.top().first 
		\State $currT \gets currT + nextWeight$ 
		\If{$currT \geq hApprox \times L$}
			break 
		\EndIf
		\State $curr \gets edgeHeap$.top().second.first 
		\State $currSet.insert(curr)$ 
		\State $visited[curr] \gets true$ 
		\State $edgeHeap$.pop() 
		\State $U$.clear() 
		\State $U \gets$ all unvisited vertices adjacent to $cur$ 
		\ForAll{ vertex $v$ in $U$}
			\State $edgeHeap.push((hApprox \times dist(v,curr)) + (ASP\times w[curr]) , (v,curr))$ 
		\EndFor
	\EndWhile

	\State $cleanTour(currSet)$
\algstore{pag}
\end{algorithmic}
\end{algorithm}

\begin{algorithm}
\begin{algorithmic}
\algrestore{pag}
	
	\State Tour $currTour$ 
	\State $tempSet \gets currSet$ 
	\State $cycleWeight , currTour \gets TSPCircuit(tempSet)$
	\ForAll{ sensor $s$ in $currTour$}
		$cycleWeight \gets cycleWeight + ASP*w[s]$
	\EndFor
	
	\While{$\neg$ edgeHeap.empty() and cycleWeight $<$ $L$}
		\State $curr \gets edgeHeap$.top().second.first 
		\State tempSet.insert(curr);
		\State $cleanTour(tempSet)$
		\State $cycleWeight, currTour \gets TSPCircuit(tempSet)$ 
		\ForAll{ sensor $s$ in $currTour$}
			$cycleWeight \gets cycleWeight + ASP*w[s]$
		\EndFor
		\If{$cycleWeight$ $>$ $L$}
			\State break 
		\EndIf
		\State currSet.insert($curr$);
		\State $visited[curr] \gets true$ 
		\State edgeHeap.pop() 
		\State $U$.clear() 
		\State $U  \gets$ all unvisited vertices adjacent to $cur$ 
		\ForAll{ vertex $v$ in U}
			\State $edgeHeap.push((hApprox \times dist(v,curr)) + (ASP\times w[curr]) , (v,curr))$ 
		\EndFor
	\EndWhile
	
	\State $S$.insert($currTour$) 

	\While{$\neg$ heap.empty()}
		\State $edge$ $e$ = $edgeHeap$.top().second;
		\State $boundary$.push($e$.second);
		\State $edgeHeap$.pop();
	\EndWhile
	
	\If{$boundary$.empty()}
		\State return false 
	\EndIf

\EndWhile
\EndFunction

\Function{cleanTour}{$vSet$ : the set of vertices in the current tour}
	\Comment{Delete all Steiner vertices from $vSet$, whose all adjacent verices are in $vSet$ itself.}
\EndFunction

\end{algorithmic}
\end{algorithm}

\pagebreak


\subsection{Scalability restriction and Minimum Latency}
Consider the extreme case of low latency requirement, when 1 MULE is assigned to each edge of the Steiner tree. This must be the smallest latency supportable with this heuristic. Lets call this latency $L_{min}$. Let $w_{max}$ be the larget weight among the wieghts of the edges of the Steiner tree. Then, $L_{min} \ge w_{max}$.
For any edge $(p,q)$ in the current Steiner tree, suppose upon recording the dfs sequence from the base station in the Steiner tree, $p$ appears before $q$; then $p$ is said to be near vertex, and $q$ is said to be the farther vertex. Then, for the case when $t_{LS}$ is not ignored, let $(i,j)$ be the maximum weighted edge with weight $w_{max}$. Also, let j be the farther 
vertex. Then, $L_{min} \ge w_{max}+t_{LS}[j]$.

Consider a single MULE collecting and aggregating all the data from all the sensor networks, and uploading it to the base station. For any sensor distribution, let the time taken by the MULE to upload the data be $T_{MBS}$. Now we will account for $t_{MBS}$. Consider the subtour containing the base station. The MULE assigned to this subtour is responsible for indirectly collecting and aggregating the data from all other subtours and delivering it to the base station. For this, it sends/recieves from other MULEs and the base station for total of $2 \times T_{MBS}$ units of time (simplistic assumption, neglecting time spent in P2P protocol between MULEs). Clearly, for any other subtour, its assigned MULE can not have a $t_{MBS}$ greater than $2 \times T_{MBS}$. So, if $T_{MBS}$ is significant contribution to total time taken by a MULE to cover its subtour, $L_{min} \ge w_{max}+t_{LS}[j]+2 \times T_{MBS}$.
