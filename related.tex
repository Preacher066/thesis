\chapter{Related Works}
Most notable related work is \cite{sim}. Their goal is to find the trajectories of $k$ MULEs to collect data from the $N$ wireless sensor nodes deployed on 2-D Euclidean space such that the data collection latency (the length of the longest trajectory among k trajectories) is minimized. They consider two cases: 
\begin{description}
\item[Case 1:] Each MULE is connected to the sink only from their original positions (base stations).
\item[Case 2:] Each MULE is connected to the sink directly at any time from any location.
\end{description}
Intuitively, it may be better to utilize MULE to MULE communication too for data transfer, which is absent from their approach, and is essential in our approach. They solve the problem by formulating k-traveling salesperson problem with neighborhood (k-TSPN) and k-rooted path cover problem with neighborhood (k-PCPN) problems. Their algorithm has a constant factor approximation, however, as they say in their section 5, their algorithm itself is based on another approximation algorithm \cite{supportSim}, which is a very complicated rounding-based algorithm and difficult to use in some occasions. Therefore, they propose heuristics to solve it.

A similar problem of covering a set of points in a plane using multiple robots \cite{roboPlan} also seemed promising, but, we were unable to accomodate the additional constraints listed above (they use Clique cover problem \cite{minCliquePartition}, which has its own application in data collection algorithms for dense sensor networks).

Another related work \cite{sim4} is on SenCars \cite{sencar}.Their algorithm takes as input $m$ the number of MULEs (SenCars), and the sensor network (sensor positions in the field). They first divide the sensor network into "polling points", then compute its Eulcidean MST. They put weights on polling points which estimates the time a SenCar will take to cover that point. The weight of a subtree is the sum of weights of its vertices and the sum of weights of its edges. Their aim is to divide the Euclidean MST into $m$ more or less equal weight subtrees, which, they argue, is a good estimate of the time a Sencar will take on a data collection tour of that subtree. They solve using ILP and also provide a greedy heuristic.

The problem with graph partitioning approach is that the rendezvous of two neighbouring MULEs is not taken in account in the TSP evaluation of a partition. In \cite{sim4}, they assume that each SenCar can forward the gathered data to one of the nearby SenCars when they move close enough. Which is reasonable because MULEs transfer data at high speed and they aggregate data before transmitting. This facilitates a MULE to transfer/recieve required data within the brief contact time.

This approach on some Euclidean MST may lead to an isolated partition, whose MULE cannot communicate with any other partition. Our approach ensures connectivity through ensuring MULE rendezvous at some point in the plane.

\cite{sim3} is a connectivity restoration algorithm for a sparse sensor network. Its authors use both Stationary Relay nodes and Mobile data collectors, or mobile relay nodes (similar to MULEs) to restore connectivity in a set of sensor islands as their sensor network. They call each island a partition. Although RNs are best for providing stable link from cluster to the sink, there is only a limited number of stationary RNs they can use. For each partition they choose a representative sensor node, which has the responsilbility to collect and store the data from its partition. For these representative nodes, they compute the Steiner tree, for the determination of positions of relay nodes. Since stationary RNs are preferable to MDC's, initially all the relay nodes are mobile (i.e. are MDCs). Each of them visits a subset of representative nodes under bounded latency L. One by one they start connecting each partition to the sink through relay nodes, and at each step they check whether the remaining MDCs are still able to cover the remaining clusters under the bound latency.

Another work that uses Steiner tree to plan MULE path is \cite{rendezvous}. Their work is applicable only on dense sensor networks, because they use RPs (rendezvous points) situated at steiner points to collect and buffer data from nodes farther away from the bases station for the MULE to visit and collect later. Ofcourse this means they have to be in range of their adjacent snesor in the Steiner tree, which can not be guarantedd in a sparse WSN.