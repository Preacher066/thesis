\chapter{Related Works}
Most notable related work is~\cite{sim}. Their goal is to find the trajectories of $k$ MULEs to collect data from the $N$ wireless sensor nodes deployed on 2-D Euclidean space such that the data collection latency (the length of the longest trajectory among k trajectories) is minimized. They consider two cases: 
\begin{description}
\item[Case 1:] Each MULE is connected to the sink only from their original positions (base stations).
\item[Case 2:] Each MULE is connected to the sink directly at any time from any location.
\end{description}
Intuitively, it may be better to utilize MULE to MULE communication too for data transfer, which is absent in their approach but essential to our approach. They solve the problem by formulating $k$-traveling salesperson problem with neighborhood ($k$-TSPN) and $k$-rooted path cover problem with neighborhood (k-PCPN) problems. Their algorithm has a constant factor approximation, however, as they say in section~5 of their paper, this algorithm itself is based on another approximation algorithm~\cite{supportSim}, which is a very complicated rounding-based algorithm and difficult to use. Therefore, they propose heuristics to solve it.

A similar problem of covering a set of points in a plane using multiple robots~\cite{roboPlan} also seemed promising, but, we were unable to accomodate the additional constraints listed above (they use Clique cover problem~\cite{minCliquePartition}, which has its own application in data collection algorithms for dense sensor networks).

Another related work~\cite{sim4} is on SenCars~\cite{sencar} (a MULE). Their algorithm takes as input $m$ the number of SenCars, the sensor network (sensor node positions in the field) and a set of polling points $P$ (positions from where the SenCar can stop and communicate with sensors around it). Their aim is to divide the set of polling points $P$ into subsets $P_1, P_2, \ldots, P_m$, each of which have approximately equal weight. The weight of a subset of polling points is the sum of operation times of a SenCar at each polling point (The time that a SenCar will take to collect data from the sensors covered by that polling point, which is directly proportional to the number of sensors covered by it), plus the time it will take the SenCar to visit each polling point in this partition. They show an integer linear programming formulation of the problem, and also provide a heuristic.

A drawback in any graph partitioning approach is that the TSP evaluation of a partition (the length of tour that the SenCar handling that partition will have to take), does not take into account the rendezvous of two neighbouring MULEs (MULEs covering neighbouring partions). Here, the authors simply assume that each SenCar can forward the gathered data to te neighbouring SenCars when they move close enough. It is a reasonable assumption in a relatively denser sensor network (where MULEs' tours can be close enough to communicate to each other) because MULEs transfer data at high speed and they aggregate data before transmitting. This facilitates a SenCar to transfer/recieve required data within the brief contact time. But in the case of a sparse sensor network containing sensor islands, each with similar weights, this approach may lead to isolated partitions as the islands may be far away. This will lead to the situation where a SenCar cannot communicate with any other SenCar, because its tour never gets close enough to another SenCar's tour. Our approach ensures connectivity through ensuring MULE rendezvous at some point in the plane.

Reference~\cite{sim3} is a connectivity restoration algorithm for a sparse sensor network. Its authors use both stationary Relay Nodes (RN) and Mobile data collectors, or mobile relay nodes (similar to MULEs) to restore connectivity in a set of sensor islands as their sensor network. They call each island a partition. Although RNs are best for providing stable link from cluster to the sink, there is only a limited number of stationary RNs they can use. For each partition they choose a representative sensor node, which has the responsilbility to collect and store the data from its partition. For these representative nodes, they compute the Steiner tree, for the determination of positions of relay nodes. Since stationary RNs are preferable to MDC's, initially all the relay nodes are mobile (i.e. are MDCs). Each of them visits a subset of representative nodes under bounded latency L. One by one they start connecting each partition to the sink through relay nodes, and at each step they check whether the remaining MDCs are still able to cover the remaining clusters under the bound latency.

Another work that uses Steiner tree to plan MULE path is~\cite{rendezvous}. Their work is applicable only on dense sensor networks, because they use RPs (rendezvous points) situated at steiner points to collect and buffer data from nodes farther away from the bases station for the MULE to visit and collect later. Ofcourse this means they have to be in range of their adjacent snesor in the Steiner tree, which can not be guaranted in a sparse WSN.


%The greedy heuristic:
%First, MST of polling points made
%w(v) is the weight of the subtree rooted at vertex v.
%We use a variable m to represent the
%number of currently available SenCars in each stage of the algorithm and initialize it to N_k. To build a subtree in each %iteration, first find a farthest leaf vertex v onT with the minimum
%weight. If w(v) < W_T(weight of tree T)/m, find its parent vertex on T, denoted by PA(v), and let v =PA(v). Check its weight %and repeat this up-tracing process until w(v) ≥W_T/m. Record
%this vertex v and consider it as the root of subtreet. All the
%vertices on t will belong to the same part and are removed from
%T, which means that the corresponding selected polling points
%on t will be assigned to the same region (or a SenCar). After
%that, update T, W_T, m and w(v') for each vertex v' on the updated
%T and repeat this operation. When there is only one available
%SenCar left, all the remaining selected polling points and their
%associated sensors are assigned to this SenCar and the procedure terminates.